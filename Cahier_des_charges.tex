\documentclass{article}
\title{Rapport du Sprint 1}
\date{11/12/2025}
\author{Groupe F}
\usepackage{amsmath, amssymb}
\usepackage{xcolor}
\usepackage{listings}
\usepackage{cancel}
\usepackage[T1]{fontenc}
\usepackage{microtype}
\usepackage{pgfgantt}
\usepackage{xcolor}
\usepackage[utf8]{inputenc}
\usepackage{lmodern}
\usepackage[a4paper]{geometry}
\usepackage{graphicx}
\usepackage{hyperref}
\usepackage{indentfirst}
\geometry{hmargin=2.5cm,vmargin=1cm}
\usepackage{babel}
\usepackage[]{tcolorbox}
\usepackage[]{enumitem}
\usepackage[]{lipsum}
\usepackage[]{multicol}

\usepackage{indentfirst}
\usepackage{hyperref}
\usepackage{graphicx}
%%\input{/home/triplaqs/Documents/Perso/Code/LaTeX/macros.tex}
%% PRINCIPALES COMMANDES DE LA MACRO
%% \int      :   intégrales
%% \sum      :   sommes
%% \mat \det :   matrices / déterminant
%% \vec      :   vecteurs
%% \klig     :   k lignes
%% \acc      :   accolades  
%% \sgt      :   intervales/accolades
%% \lim      :   limites
%% \blu \gre :   couleur 
%% \mat \cur :   police
\begin{document}
	\maketitle	
	%encadré qui rend bizarre :	
	%\begin{tcolorbox}[colback=blue!5!white,colframe=black!75!black]
	%\gauchedroite{Groupe F}{}\par
	%\begin{center}
    %\LARGE{\textbf{Rapport du premier Sprint}}
	%\end{center}
	%\end{tcolorbox}

   \section{Compte rendu de la réunion avec l'encadrant}

   Nous avons pris, dans un premier temps, rendez-vous avec Marc HARTLEY, notre encadrant, 
   le jeudi 11 décembre à 13h15 afin d'éclaircir les objectifs et les attentes de ce projet.\\
   Lors de cet entretien étaient présents MILEZI Gaëlle, DELRIEU Saurane, PEREYROL Axel et BARRAUX Satine. \\
   Afin de pouvoir nous guider au mieux, Monsieur HARTLEY nous a préalablement demandé de 
   préparer ce que nous avions compris du sujet et ce que nous voulions faire dans ce projet, 
   en quoi il nous intéressait et enfin ce que nous voudrions faire si nous avions un temps de travail infini.\\
   À la suite de ce premier échange, nous avons éclairci les détails sur la mise au point, notamment avec l'aide d'un
   exemple, décidé de ce que nous ferions durant les prochains mois et établi un premier aperçu de l'objectif que nous
   voulions atteindre.\\
   Points éclaircis et décisions prises durant cette réunion :
   \begin{itemize}
      \item Le projet doit être intéractif.
      \item Il doit y avoir une interface graphique.
      \item Il doit y avoir une simulation de fluide.
      \item Le languange de programmation sera le C++.
      \item L'interface graphique sera faite avec OpenGL.
      \item Le projet sera fait en 2D.
      \item Décision des fonctionnalités principales (MVP).
      \item Different module du projet.
   \end{itemize}
   
   \section{Cahier de charges}
   
   \subsection{Description du sujet et des objectifs}

   \indent Intéressons nous maintentant au projet en lui même. 
   Nous avons comme projet de créer une interface (cf \hyperref[schéma]{schéma}) sur laquelle nous pourrons voir le comportement du fluide 
   et une barre de commandes (types curseurs) permettant d'intéragir avec le fluide. \\
   Nous avons décidé dans un premier temps de nous concentrer 
   sur la simulation de l'eau. Nous allons modéliser ce fluide en faisant varier plusieurs paramètres tels que la pression, la température ou encore la viscosité.\\
   Une fois la première étape réussie nous avons envisager des extensions plus complexes telles que le contrôle à distance via un smartphone ou l'interaction entre le fluide 
   et de la musique. 

   \newpage

   \begin{figure}[h] 
      \centering
      \includegraphics[width=0.7\textwidth]{PJ_proj/schemaproj2.png}
      \label{schéma}
      \caption{Schéma simple du projet}
   \end{figure}

   \subsection{Périmètre retenu pour le projet}

   Le périmètre retenu pour le projet est une approche eulérienne du fluide en 2D. 
   Nous utiliserons des vecteurs sur une grille, ainsi que des curseurs. 
   Nous utiliserons notamment des équations sur les fluides afin d'avoir un projet 
   interactif qui permettra une visualisation et une interaction en temps réel 
   (notamment via l'utilisation de la souris ou, plus tard, d'un second écran). 
   L'objectif est d'arriver, d'ici la fin de l'année, à un fluide pouvant interagir 
   avec le son et pouvant être contrôlé à distance via un smartphone. Nous
   utiliserons également des espaces vectoriels afin de modéliser au mieux le mouvement 
   du fluide. Il sera également important de faire attention au côté 
   esthétique de l'interface graphique en y ajoutant des éléments visuels attrayants afin que celle ci 
   soit des plus ludique.


   \subsection{Fonctionnalités principales (MVP)}

   Les fonctionnalités principales que nous allons implémenter sont :
   \begin{itemize}
      \item Une interface graphique avec une fenêtre d'affichage du fluide et des curseurs.
      \item La simulation d'un fluide (l'eau) en 2D.
      \item La possibilité d'agir sur le fluide à l'aide des curseurs.
      \item La possibilité d'ajouter des obstacles au fluide.
      \item Une musique faisant vibrer le son.
      \item La possibilité de contrôler le fluide à distance via un smartphone.
   \end{itemize}

   \subsection{Contraintes techniques et choix de technologies}

   L'ensemble du projet sera programmé en C++ avec une interface graphique OpenGL.
   De plus, nous procéderons sur une grille de $100 \times 100$ pour la simulation du fluide et 
   la fréquence d'images sera de 10 images par secondes au minimum.
	
   \section{Première architecture du projet}
   
   \subsection{Organisation générale}\label{partie 3.1}
   Notre projet sera structuré en plusieurs modules :

   \begin{itemize}
      \item \textbf{Module de gestion des paramètres :} \\
      Ce module permet de regrouper les réglages de la simulation 
      (taille de la grille, viscosité). \\
      Classe envisagée : \fbox{\texttt{SimulationParametres}}
      

      \item \textbf{Module de calcul pour la simulation du fluide :}\\
      Dans ce module nous implémenterons les étapes principales de la
      simulation du fluide : advection, diffusion et projection.\\
      Classe envisagée : \fbox{\texttt{EquationFluide}}


      \item \textbf{Module d'affichage et de visualisation :}\\
      Ce module permettra l'affichage des différentes simulations
      (affichage du champ scalaire, affichage du champ vectoriel)\\
      Classe envisagée : \fbox{\texttt{Affiche}}


      \item \textbf{Module d'interaction avec l'utilisateur :}\\
      Ce module permettra de gérer les interactions (clavier, souris)
      avec l'utilisateur. Nous pourrons ajouter des forces ou des obstacles.
      Classe envisagée : \fbox{\texttt{GestionInput}}



      \item \textbf{Module principal :}\\
      Ce module sera le point d'entrée du programme, il met en lien tous les modules.
      C'est-à-dire qu'il récupère les entrées de l'utilisateur, met à jour le fluide,
      et affiche. 

   \end{itemize}
   

   
   
   \subsection{Schéma simple/Description textuelle}

   Voilà une représentation de l'architecture et du fonctionnement de notre projet.
   \begin{figure}[h] 
      \centering
      \includegraphics[width=0.7\textwidth]{PJ_proj/schema_fonctionnement.png}
      \caption{Schéma simple architecture/fonctionnement du projet}
   \end{figure}

   

   

   
   
   \section{Organisation interne du groupe}

   Notre groupe est composé de quatre étudiants de double-licence. Pour mener au mieux le projet, 
   nous avons décidé de nous organiser de la manière suivante.

   Premièrement nous avons choisi de donner des rôles précis à chaque membre 
   (cf. \hyperref[partieroles]{partie répartition des rôles}). Ces rôles sont susceptibles
   de changer en fonction de l'envie des membres, des besoins du groupe et des compétences de chacun. \\
   Ensuite, nous avons convenu, avec notre chargé de projet, d'un rendez-vous hebdomadaire. Cela nous 
   permettra de ne pas trop nous écarter de notre idée et de valider nos avancées sur le projet.
   A la suite de cette réunion, nous, les membres du groupe, nous mettrons d'accord sur les différentes 
   tâches à venir et nous répartirons les rôles. \\
   De plus, pour nous permettre d'avancer au mieux le projet en prenant en compte les contraintes de 
   chacun nous avons décidé que chaque membre du groupe travaillerait sur sa propre branche git. Cela permet à
   chacun d'avancer à son rythme sur sa tâche. Une fois que tous les membres seront d'accord sur les
   modifications de chacun, nous mettrons en commun sur la branche principale. \\

	\subsection{Répartition des rôles}\label{partieroles}
   Nous avons décidé, pour une progression optimale, que chaque membre serait en charge d'un module.
   Le développement du module principal sera réalisé collectivement. 

   \subsubsection*{Gaëlle Milezi :}
   Gaëlle sera en charge du module calcul pour la simulation du fluide.
   Elle devra implémenter les différents algorithmes pour permettre la simulation 
   (advection, diffusion, projection).
   
   \subsubsection*{Satine Barraux :}
   Satine s'occupera du module de la gestion des paramètres. 
   Elle sera responsable de la définition et du contrôle des paramètres.

   \subsubsection*{Saurane Delrieu :}
   Saurane sera en charge du modue interactions. 
   Son rôle est de traduire les actions (clavier ou souris) de l'utilisateur 
   en modifications du champ de fluide.

   \subsubsection*{Axel Pereyrol :}
   Axel sera responsable du module d'affichage.
   Il assure le rendu graphique en temps réel, la clarté et la cohérence de la visualisation.
   
   

   \subsection{Outils utilisés}
   Pour mener à bien ce projet, nous avons mis en place un git (sur GitHub) auquel notre chargé de projet a accès.
   De plus, comme dit plus haut, nous avons instauré deux réunions hebdomadaires, celles-ci nous permettront d'avancer régulièrement le projet.\\
   Nous avons également mis en place un groupe discord sur lequel les quatre membres du groupe et notre chargé de projet sont présents. Celui-ci nous 
   aide à communiquer et échanger les informations à distance.\\
   Finalement, nous avons choisi de développer ce projet en C++ (avec OpenGL) cela nous permettra d'avoir une simulation performante.

   \section{Planning prévisionnel}
   
   \subsection{Découpage en étapes}
   Afin d’assurer une progression régulière, le travail sera découpé en plusieurs étapes correspondant à des jalons fonctionnels clairement identifiés.

   \begin{itemize}
    
    \item \textbf{Phase 1 – Implémentation de la simulation (janvier-février)} \\
    L’objectif est d’obtenir une simulation de fluide 2D fonctionnelle sur une grille régulière. 
    Cette phase inclut la représentation des champs physiques (vitesse, densité, pression) et l’implémentation des 
    principales étapes de calcul : advection, diffusion et projection afin de garantir l’incompressibilité du fluide.
    
    \item \textbf{Phase 2 – Interaction et visualisation (fevrier-mars)} \\
    Cette étape vise à enrichir la simulation par des interactions utilisateur (ajout de forces locales, 
    injection de matière) et par une visualisation plus avancée du fluide sous forme d’animation 2D en temps réel.
    
    \item \textbf{Phase 3 – Extensions et finalisation (mars-avril)} \\
    Cette dernière phase est dédiée à l’implémentation de fonctionnalités optionnelles (contrôle à distance via 
    smartphone, réactions au son/à la musique), ainsi qu’à la finalisation du rapport et de la présentation.
   \end{itemize}


   \subsection{Priorités des premiers Sprints}
   Les premiers sprints, correspondant principalement aux mois de janvier et février, sont centrés sur la mise en 
   place de fondations solides pour le projet. Les priorités identifiées sont les suivantes :

   \begin{itemize}
      \item Compréhension et formalisation du modèle de simulation du fluide.
      \item Mise en place d’une architecture de code en C++.
      \item Création d’une première visualisation graphique du fluide.
      \item Obtention d’une simulation stable d’un fluide de type eau avec des paramètres simples.
   \end{itemize}

   Ces objectifs constituent un socle fonctionnel minimal sur lequel pourront s’appuyer les développements ultérieurs 
   et les extensions du projet.

   \subsection{Tâches prévues de janvier à avril (Gantt prévisionnel)}

   Les principales tâches prévues pour la durée du projet sont représentées sous la forme
   d’un diagramme de Gantt prévisionnel, de la dernière semaine de janvier à la fin du mois d’avril.

   \noindent
   \makebox[\textwidth][c]{
   \begin{ganttchart}[
      hgrid,
      vgrid,
      x unit=0.9cm,
      bar height=0.6,
      group height=0.8
   ]{1}{14}

   \gantttitle{Planning prévisionnel du projet}{14} \\
   \gantttitlelist{1,...,14}{1} \\

   % =====================
   % JANVIER / FEVRIER
   % =====================
   \ganttgroup{Simulation du fluide}{1}{5} \\

   \ganttbar[
      bar/.append style={fill=green}
   ]{Advection}{1}{2} \\

   \ganttbar[
      bar/.append style={fill=blue}
   ]{Diffusion et viscosité}{2}{4} \\

   \ganttbar[
      bar/.append style={fill=red}
   ]{Projection et incompressibilité}{3}{5} \\

   \ganttbar[
      bar/.append style={fill=orange}
   ]{Tests de stabilité et réglages}{4}{5} \\

   % =====================
   % MARS
   % =====================
   \ganttgroup{Interaction et visualisation}{6}{9} \\

   \ganttbar[
      bar/.append style={fill=red}
   ]{Forces et interactions utilisateur}{6}{7} \\

   \ganttbar[
      bar/.append style={fill=green}
   ]{Paramètres dynamiques du fluide}{7}{8} \\

   \ganttbar[
      bar/.append style={fill=blue}
   ]{Visualisation}{7}{9} \\

   \ganttbar[
      bar/.append style={fill=orange}
   ]{Contrôle via smartphone}{8}{9} \\

   % =====================
   % AVRIL
   % =====================
   \ganttgroup{Extensions}{10}{14} \\

   \ganttbar[
      bar/.append style={fill=blue}
   ]{Finalisation des interactions}{10}{11} \\

   \ganttbar[
      bar/.append style={fill=red}
   ]{Analyse des performances et stabilité}{10}{14} \\

   \ganttbar[
      bar/.append style={fill=green}
   ]{Mode musique : réaction au son}{11}{13} \\

   \ganttbar[
      bar/.append style={fill=purple}
   ]{Intégration finale et rédaction du rapport}{12}{14}

   \end{ganttchart}
   }

   \textbf{Répartition des tâches :}
   \begin{itemize}
      \item \textcolor{blue}{Axel}
      \item \textcolor{red}{Saurane}
      \item \textcolor{green}{Gaëlle}
      \item \textcolor{orange}{Satine}
      \item \textcolor{purple}{Tout le monde}
   \end{itemize}

   \section{Prototype réalisé}

   \subsection{Description du prototype actuel}

   À l’issue de la première semaine de projet, un prototype minimal a été développé afin de valider 
   les choix techniques initiaux. Ce prototype comprend :

   \begin{itemize}
      \item Une grille 2D................
      \item blabla un triangle !
      \item Un affichage graphique simple permettant de visualiser.............
      \item Des tests ?? on peut pas faire 1000 trucs en 1 semaine non plus !
   \end{itemize}

   Ce prototype, bien que très simple, permet de vérifier la faisabilité de la simulation et constitue une 
   base de travail pour les développements ultérieurs.

   \subsection{Justification des choix initiaux}

   Le langage \textbf{C++} a été retenu pour ce projet en raison de ses performances et de son contrôle précis 
   de la mémoire, des aspects essentiels pour une simulation numérique interactive en temps réel.

   Le choix d’une \textbf{approche sur grille (Eulerienne)} a été privilégié par rapport à une approche 
   par particules, car elle est plus simple à mettre en œuvre dans un premier temps et mieux adaptée à la 
   visualisation continue de phénomènes tels que l’eau ou la fumée.

   Le projet débute avec la simulation d’un fluide de type eau, avec des paramètres ajustables (pression, 
   température, viscosité), afin de valider progressivement le modèle physique avant d’envisager des extensions 
   plus complexes, telles que l’interaction avec la musique ou le contrôle à distance via un smartphone.

\end{document}

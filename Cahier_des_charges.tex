\documentclass{article}
\title{Rapport du Sprint 1}
\date{11/12/2025}
\author{Groupe F}
\usepackage{amsmath, amssymb}
\usepackage{xcolor}
\usepackage{listings}
\usepackage{cancel}
\usepackage[T1]{fontenc}
\usepackage{microtype}
\usepackage[utf8]{inputenc}
\usepackage{lmodern}
\usepackage[a4paper]{geometry}
\geometry{hmargin=2.5cm,vmargin=1cm}
\usepackage{babel}
\usepackage[]{tcolorbox}
\usepackage[]{enumitem}
\usepackage[]{lipsum}
\usepackage[]{multicol}
%%\input{/home/triplaqs/Documents/Perso/Code/LaTeX/macros.tex}
%% PRINCIPALES COMMANDES DE LA MACRO
%% \int      :   intégrales
%% \sum      :   sommes
%% \mat \det :   matrices / déterminant
%% \vec      :   vecteurs
%% \klig     :   k lignes
%% \acc      :   accolades  
%% \sgt      :   intervales/accolades
%% \lim      :   limites
%% \blu \gre :   couleur 
%% \mat \cur :   police
\begin{document}
	\maketitle	
	%encadré qui rend bizarre :	
	%\begin{tcolorbox}[colback=blue!5!white,colframe=black!75!black]
	%\gauchedroite{Groupe F}{}\par
	%\begin{center}
    %\LARGE{\textbf{Rapport du premier Sprint}}
	%\end{center}
	%\end{tcolorbox}

   \section{Compte rendu de la réunion avec l'encadrant}
   
   \subsection*{Date et heure de la réunion}
   \subsection*{Participant.es}
   \subsection*{Points clarifiés et décisions prises}
   
   \section{Cahier de charges}
   
   \subsection{Description du sujet et des objectifs}
   \subsection{Périmètre retenu pour le projet}
   \subsection{Fonctionnalités principales (MVP)}
   \subsection{Contraintes techniques et choix de technologies}
	
   \section{Première architecture du projet}
   
   \subsection{Organisation générale}
   \subsection{Schéma simple/Description textuelle}
   
   \section{Organisation interne du groupe}
	\subsection{Répartition des rôles}
   \subsubsection*{Gaëlle Milezi :}
   \subsubsection*{Saurane Delrieu :}
   \subsubsection*{Satine Barraux :}
   \subsubsection*{Axel Pereyrol :}
   \subsection{Outils utilisés}

   \section{Planning prévisionnel}
   
   \subsection{Découpage en étapes}
   Afin d’assurer une progression régulière, le travail sera découpé en plusieurs étapes correspondant à des jalons fonctionnels clairement identifiés.

   \begin{itemize}
    
    \item \textbf{Phase 1 – Implémentation de la simulation (janvier-février)} \\
    L’objectif est d’obtenir une simulation de fluide 2D fonctionnelle sur une grille régulière. 
    Cette phase inclut la représentation des champs physiques (vitesse, densité, pression) et l’implémentation des 
    principales étapes de calcul : advection, diffusion et projection afin de garantir l’incompressibilité du fluide.
    
    \item \textbf{Phase 2 – Interaction et visualisation (fevrier-mars)} \\
    Cette étape vise à enrichir la simulation par des interactions utilisateur (ajout de forces locales, 
    injection de matière) et par une visualisation plus avancée du fluide sous forme d’animation 2D en temps réel.
    
    \item \textbf{Phase 3 – Extensions et finalisation (mars-avril)} \\
    Cette dernière phase est dédiée à l’implémentation de fonctionnalités optionnelles (contrôle à distance via 
    smartphone, réactions au son/à la musique), ainsi qu’à la finalisation du rapport et de la présentation.
   \end{itemize}


   \subsection{Priorités des premiers Sprints}
   Les premiers sprints, correspondant principalement aux mois de janvier et février, sont centrés sur la mise en 
   place de fondations solides pour le projet. Les priorités identifiées sont les suivantes :

   \begin{itemize}
      \item Compréhension et formalisation du modèle de simulation du fluide.
      \item Mise en place d’une architecture de code en C++.
      \item Création d’une première visualisation graphique du fluide.
      \item Obtention d’une simulation stable d’un fluide de type eau avec des paramètres simples.
   \end{itemize}

   Ces objectifs constituent un socle fonctionnel minimal sur lequel pourront s’appuyer les développements ultérieurs 
   et les extensions du projet.

   \subsection{Tâches prévues de janvier au 30 avril (Gantt prévisionnel)}

   Les principales tâches prévues pour la durée du projet sont réparties comme suit :
   
   \par\vspace{0.5\baselineskip}
   \textbf{Février}
   \begin{itemize}
      \item Implémentation de l’advection, diffusion (viscosité), projection
      \item Tests de stabilité en fonction des paramètres choisi
      \item Premières simulations visuelles du fluide
   \end{itemize}

   \textbf{Mars}
   \begin{itemize}
      \item Ajout des interactions utilisateur (forces locales, injection de matière)
      \item Paramétrage dynamique du fluide (pression, température, viscosité)
      \item Amélioration de la visualisation (couleurs, animations)
      \item Début de l’implémentation du contrôle à distance via smartphone
   \end{itemize}

   \textbf{Avril}
   \begin{itemize}
      \item Finalisation du contrôle à distance
      \item Implémentation d’un prototype du "mode musique"
      \item Tests finaux et corrections
      \item Rédaction du rapport final et préparation de la soutenance
   \end{itemize}

   La répartition des tâches entre les membres du groupe est prévue de manière équilibrée, chaque étudiant étant 
   responsable d’un axe principal du projet tout en collaborant sur les autres aspects du projet.


   \section{Prototype réalisé}

   \subsection{Description du prototype actuel}

   À l’issue de la première semaine de projet, un prototype minimal a été développé afin de valider 
   les choix techniques initiaux. Ce prototype comprend :

   \begin{itemize}
      \item Une grille 2D................
      \item blabla un triangle !
      \item Un affichage graphique simple permettant de visualiser.............
      \item Des tests ?? on peut pas faire 1000 trucs en 1 semaine non plus !
   \end{itemize}

   Ce prototype, bien que très simple, permet de vérifier la faisabilité de la simulation et constitue une 
   base de travail pour les développements ultérieurs.

   \subsection{Justification des choix initiaux}

   Le langage \textbf{C++} a été retenu pour ce projet en raison de ses performances et de son contrôle précis 
   de la mémoire, des aspects essentiels pour une simulation numérique interactive en temps réel.

   Le choix d’une \textbf{approche sur grille (Eulerienne)} a été privilégié par rapport à une approche 
   par particules, car elle est plus simple à mettre en œuvre dans un premier temps et mieux adaptée à la 
   visualisation continue de phénomènes tels que l’eau ou la fumée.

   Le projet débute avec la simulation d’un fluide de type eau, avec des paramètres ajustables (pression, 
   température, viscosité), afin de valider progressivement le modèle physique avant d’envisager des extensions 
   plus complexes, telles que l’interaction avec la musique ou le contrôle à distance via un smartphone.

\end{document}

\documentclass{article}
\title{Rapport du Sprint 1}
\date{11/12/2025}
\author{Groupe F}
\usepackage{amsmath, amssymb}
\usepackage{xcolor}
\usepackage{listings}
\usepackage{cancel}
\usepackage[T1]{fontenc}
\usepackage[utf8]{inputenc}
\usepackage{lmodern}
\usepackage[a4paper]{geometry}
\geometry{hmargin=2.5cm,vmargin=1cm}
\usepackage{babel}
\usepackage[]{tcolorbox}
\usepackage[]{enumitem}
\usepackage[]{lipsum}
\usepackage[]{multicol}

\usepackage{indentfirst}
\usepackage{hyperref}
\usepackage{graphicx}
%%\input{/home/triplaqs/Documents/Perso/Code/LaTeX/macros.tex}
%% PRINCIPALES COMMANDES DE LA MACRO
%% \int      :   intégrales
%% \sum      :   sommes
%% \mat \det :   matrices / déterminant
%% \vec      :   vecteurs
%% \klig     :   k lignes
%% \acc      :   accolades  
%% \sgt      :   intervales/accolades
%% \lim      :   limites
%% \blu \gre :   couleur 
%% \mat \cur :   police
\begin{document}
	\maketitle	
	%encadré qui rend bizarre :	
	%\begin{tcolorbox}[colback=blue!5!white,colframe=black!75!black]
	%\gauchedroite{Groupe F}{}\par
	%\begin{center}
    %\LARGE{\textbf{Rapport du premier Sprint}}
	%\end{center}
	%\end{tcolorbox}

   \section{Compte rendu de la réunion avec l'encadrant}
   
   \subsection*{Date et heure de la réunion}
   \subsection*{Participant.es}
   \subsection*{Points clarifiés et décisions prises}
   
   \section{Cahier de charges}
   
   \subsection{Description du sujet et des objectifs}

   \indent Intéressons nous maintentant au projet en lui même. Nous avons comme projet de créer une interface (cf \hyperref[schéma]{schéma}) sur laquelle nous pourrons voir le comportement du fluide 
   et une barre de commandes (types curseurs) permettant d'intéragir avec le fluide. \\
   Nnous avons décidé dans un premier temps de nous concentrer 
   sur la simulation de l'eau. Nous allons modéliser ce fluide en faisant varier plusieurs paramètres tels que la pression, la température ou encore la viscosité.\\
   Une fois la première étape réussie nous avons envisager des extensions plus complexes telles que le contrôle à distance via un smartphone ou l'interaction entre le fluide 
   et la musique. 

   \begin{figure}[h] 
      \label{schéma}
      \centering
      \includegraphics[width=0.7\textwidth]{schemaproj2.png}
      \caption{Schéma simple du projet}
   \end{figure}
   
   \subsection{Périmètre retenu pour le projet}
   \subsection{Fonctionnalités principales (MVP)}
   \subsection{Contraintes techniques et choix de technologies}
	
   \section{Première architecture du projet}
   
   \subsection{Organisation générale}\label{partie 3.1}
   Notre projet sera structuré en plusieurs modules :

   \begin{itemize}
      \item \textbf{Module gestion des paramètres :} \\
      Ce module permet de regrouper les réglages de la simulation 
      (taille de la grille, viscosité). \\
      Classe envisagée : \fbox{\texttt{SimulationParametres}}
      

      \item \textbf{Module calcul pour la simulation du fluide :}\\
      Dans ce module nous implémenterons les étapes principales de la
      simulation du fluide : advection, diffusion et projection.\\
      Classe envisagée : \fbox{\texttt{EquationFluide}}


      \item \textbf{Module d'affichage et de visualisation :}\\
      Ce module permettra l'affichage des différentes simulation 
      (champ de vecteurs, champ de scalaires)\\
      Classe envisagée : \fbox{\texttt{Affiche}}


      \item \textbf{Module d'interaction avec l'utilisateur :}\\
      Ce module permettra de gérer les interactions (clavier, souris)
      avec l'utilisateur. Nous pourrons ajouter des forces ou des obstacles.
      Classe envisagée : \fbox{\texttt{GestionInput}}



      \item \textbf{Module principal :}\\
      Ce module sera le point d'entrée du programme, il met en lien tous les modules.
      C'est-à-dire qu'il récupère les entrées de l'utilisateur, met à jour le fluide,
      et affiche. 

   \end{itemize}
   

   
   
   \subsection{Schéma simple/Description textuelle}

   Voilà une représentation de l'architecture et du fonctionnement de notre projet.
   \begin{figure}[h] 
      \centering
      \includegraphics[width=0.7\textwidth]{schema_fonctionnement.png}
      \caption{Schéma simple architecture/fonctionnement du projet}
   \end{figure}

   

   

   
   
   \section{Organisation interne du groupe}

   Notre groupe est composé de quatre étudiants de double-licence.
   Pour mener au mieux le projet, nous avons décidé de nours organiser de la manière suivante.\\
   Pemièrement nous avons choisi 
   de donner des rôles précis à 
   chaque membre (cf. \hyperref[partieroles]{partie répartition des rôles}). Ces rôles sont susceptibles
    de changer
   en fonction de l'envie des membres, des besoins du groupe et des compétences de chacun. \\
   Ensuite, nous avons convenu, avec notre chargé de projet, d'un rendez-vous hebdomadaire. Cela nous 
   permettra de ne pas trop
    s'écarter de notre idée et de valider nos avancées sur le projet.
   A la suite de cette réunion, nous, les membres du groupe, nous mettrons d'accord sur les différentes 
   tâches à venir et nous 
   répartirons les rôles. \\
   De plus, pour nous permettre d'avancer au mieux le projet en prenant en compte les contraintes de 
   chacun nous avons décidé 
   que chaque membre du groupe travaillerait sur sa propre branche git. Cela permet à
   chacun d'avancer à son rythme sur sa tâche. Une fois que tous les membres seront d'accord sur les
    modifications de chacun, 
   nous mettrons en commun sur la branche principale. \\

	\subsection{Répartition des rôles}\label{partieroles}
   Nous avons décidé, pour une progression optimale, que chaque membre serait en charge d'un module.
   Le développement du module principal sera réalisé collectivement. 

   \subsubsection*{Gaëlle Milezi :}
   Gaëlle sera en charge du module calcul pour la simulation du fluide.
   Elle devra implémenter les différents algorithmes pour permettre la simulation 
   (advection, diffusion, projection).
   
   \subsubsection*{Satine Barraux :}
   Satine s'occupera du module de la gestion des paramètres. 
   Elle sera responsable de la définition et du contrôle des paramètres.

   \subsubsection*{Saurane Delrieu :}
   Saurane sera en charge du modue interactions. 
   Son rôle est de traduire les actions (clavier ou souris) de l'utilisateur 
   en modifications du champ de fluide.

   \subsubsection*{Axel Pereyrol :}
   Axel sera responsable du module d'affichage.
   Il assure le rendu graphique en temps réel, la clarté et la cohérence de la visualisation.
   
   

   \subsection{Outils utilisés}
   Pour mener à bien ce projet, nous avons mis en place un git (sur GitHub) auquel notre chargé de projet a accès.
   De plus, nous avons instauré une réunion hebdomadaire avec tous les membres du groupe et notre chargé de projet, cela nous permettra de valider nos choix et 
   d'avancer constamment. 
   Nous avons également décidé de faire une réunion hebdomadaire avec seulement les membres du groupe pour nous répartir les tâches chaque semaine.\\
   Finalement, nous avons choisi de développer ce projet en C++ (avec OpenGL) cela nous permettra d'avoir une simulation performante.

   \section{Planning prévisionnel}
   
   \subsection{Découpage en étapes}
   \subsection{Priorités des premiers Sprints}
   \subsection{tâches prévues de Janvier au 30 Avril}
   \subsection*{Diagramme de Gantt}

   \section{Prototype réalisé}

	\subsection{Description des tests et développements}
	\subsection{Justification des choix}
	\subsection{Justification des choix}
\end{document}

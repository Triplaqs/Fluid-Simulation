\documentclass{article}
\title{Rapport du Sprint 1}
\date{11/12/2025}
\author{Groupe F}
\usepackage{amsmath, amssymb}
\usepackage{xcolor}
\usepackage{listings}
\usepackage{cancel}
\usepackage[T1]{fontenc}
\usepackage[utf8]{inputenc}
\usepackage{lmodern}
\usepackage[a4paper]{geometry}
\usepackage{graphicx}
\usepackage{hyperref}
\usepackage{indentfirst}
\geometry{hmargin=2.5cm,vmargin=1cm}
\usepackage{babel}
\usepackage[]{tcolorbox}
\usepackage[]{enumitem}
\usepackage[]{lipsum}
\usepackage[]{multicol}
%%\input{/home/triplaqs/Documents/Perso/Code/LaTeX/macros.tex}
%% PRINCIPALES COMMANDES DE LA MACRO
%% \int      :   intégrales
%% \sum      :   sommes
%% \mat \det :   matrices / déterminant
%% \vec      :   vecteurs
%% \klig     :   k lignes
%% \acc      :   accolades  
%% \sgt      :   intervales/accolades
%% \lim      :   limites
%% \blu \gre :   couleur 
%% \mat \cur :   police
\begin{document}
	\maketitle	
	%encadré qui rend bizarre :	
	%\begin{tcolorbox}[colback=blue!5!white,colframe=black!75!black]
	%\gauchedroite{Groupe F}{}\par
	%\begin{center}
    %\LARGE{\textbf{Rapport du premier Sprint}}
	%\end{center}
	%\end{tcolorbox}

   \section{Compte rendu de la réunion avec l'encadrant}

   Nous avons pris, dans un premier temps, rendez-vous avec Marc HARTLEY, notre encadrant, 
   le jeudi 11 décembre à 13 heures 15 afin d'éclaircir les objectifs et les attentes de ce projet.\\
   Lors de cet entretien étaient présents MILLEZI Gaëlle, DELRIEU Saurane, PEREYROL Axel et BARRAUX Satine. \\
   Afin de pouvoir nous guider au mieux, Monsieur HARTLEY nous a préalablement demandé de 
   préparer ce que nous avions compris du sujet et ce que nous voulions faire dans ce projet, 
   en quoi il nous intéressait et enfin ce que nous voudrions faire si nous avions un temps de travail infini.\\
   À la suite de ce premier échange, nous avons éclairci les détails sur la mise au point, notamment avec l'aide d'un
   exemple, décidé de ce que nous ferions durant les prochains mois et établi un premier aperçu de l'objectif que nous
   voulions atteindre.\\
   Points éclaircis et décisions prises durant cette réunion :
   \begin{itemize}
      \item Le projet doit être intéractif.
      \item Il doit y avoir une interface graphique.
      \item Il doit y avoir une simulation de fluide.
      \item Le languange de programmation sera le C++.
      \item L'interface graphique sera faite avec OpenGL.
      \item Le projet sera fait en 2D.
      \item Décidession des fonctionnalités principales (MVP).
      \item Different module du projet.
   \end{itemize}
   
   \section{Cahier de charges}
   
   \subsection{Description du sujet et des objectifs}

   \indent Intéressons nous maintentant au projet en lui même. 
   Nous avons comme projet de créer une interface (cf \hyperref[schéma]{schéma}) sur laquelle nous pourrons voir le comportement du fluide 
   et une barre de commandes (types curseurs) permettant d'intéragir avec le fluide. \\
   Nous avons décidé dans un premier temps de nous concentrer 
   sur la simulation de l'eau. Nous allons modéliser ce fluide en faisant varier plusieurs paramètres tels que la pression, la température ou encore la viscosité.\\
   Une fois la première étape réussie nous avons envisager des extensions plus complexes telles que le contrôle à distance via un smartphone ou l'interaction entre le fluide 
   et la musique. 

   \newpage

   \begin{figure}[h] 
      \centering
      \includegraphics[width=0.7\textwidth]{schemaproj2.png}
      \caption{Schéma simple du projet}
      \label{schéma}
   \end{figure}

   \subsection{Périmètre retenu pour le projet}

   Le périmètre retenu pour le projet est une approche eulérienne du fluide en 2D. 
   Nous utiliserons des vecteurs sur une grille, ainsi que des curseurs. 
   Nous utiliserons notamment des équations sur les fluides afin d'avoir un projet 
   interactif qui permettra une visualisation et une interaction en temps réel 
   (notamment via l'utilisation de la souris ou, plus tard, d'un second écran). 
   L'objectif est d'arriver, d'ici la fin de l'année, à un fluide pouvant interagir 
   avec le son et pouvant être contrôlé à distance via un smartphone. Nous
   utiliserons également des espaces vectoriels afin de modéliser au mieux le mouvement 
   du fluide. Il sera également important de faire attention au côté 
   esthétique de l'interface graphique en y ajoutant des éléments visuels attrayants afin que celle ci 
   soit des plus ludique.


   \subsection{Fonctionnalités principales (MVP)}

   Les fonctionnalités principales que nous allons implémenter sont :
   \begin{itemize}
      \item Une interface graphique avec une fenêtre d'affichage du fluide et des curseurs.
      \item La simulation d'un fluide (l'eau) en 2D.
      \item La possibilité d'agir sur le fluide à l'aide des curseurs.
      \item La possibilité d'ajouter des obstacles dans le fluide.
      \item Une musique fesant vibrer le son.
      \item La possibilité de contrôler le fluide à distance via un smartphone
   \end{itemize}

   \subsection{Contraintes techniques et choix de technologies}

   L'ensemble du projet sera programer en C++ avec une interface graphique OpenGL.
   De plus, nous procéderons sur une grille de $100 \times 100$ pour la simulation du fluide et 
   le vitsse de rafraichissement sera de 10 images par secondes au minimum.
	
   \section{Première architecture du projet}
   
   \subsection{Organisation générale}
   \subsection{Schéma simple/Description textuelle}
   
   \section{Organisation interne du groupe}
	\subsection{Répartition des rôles}
   \subsubsection*{Gaëlle Milezi :}
   \subsubsection*{Saurane Delrieu :}
   \subsubsection*{Satine Barraux :}
   \subsubsection*{Axel Pereyrol :}
   \subsection{Outils utilisés}

   \section{Planning prévisionnel}
   
   \subsection{Découpage en étapes}
   \subsection{Priorités des premiers Sprints}
   \subsection{tâches prévues de Janvier au 30 Avril}
   \subsection*{Diagramme de Gantt}

   \section{Prototype réalisé}

	\subsection{Description des tests et développements}
	\subsection{Justification des choix}
   
\end{document}
